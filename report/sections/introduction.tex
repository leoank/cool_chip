\documentclass[../main.tex]{subfiles}
\begin{document}
\section{Introduction}
\labsec{intro}

Due to increasing miniaturization and 3D integration of electronics,
the heat flux of chips has increased tremendously.
With current peak values of around $103 W/cm^2$,
we can expect this to reach upto $2300 W/cm^2$ in coming years.\sidecite{Herwig2013}.

So, a significant number of thermal engineering researchers are seeking new methods for enhancing heat transfer through fins and between fins and surrounding fluid.\sidecite{Siddique_2010}
In our work, we explored three of these heat transfer enhancement techniques.

\subsection{Optimization of fin profile}
Fins as extended surfaces can enhance the heat dissipation rate from hot surfaces owing to the enhancement in
the effectiveness of the surface to be cooled. Fins can be effective enough to cool surfaces especially when natural convection is present in process.
Erdem Cuce et al. \sidecite{Cuce2017} and H.Azarkish et al. \sidecite{AZARKISH20101938} showed that optimisation of fin profile is of vital importance in terms of the rate of heat transfer from a hot surface,
and the optimisation procedure depends on several factors.
Giampietro Fabbri\sidecite{FABBRI19972165},  Hamidreza Najafi et al. \sidecite{NAJAFI20111839} and Diego Copiello \sidecite{COPIELLO20091167} explored optimization of fin profiles using genetic algorithms.
\\
In our work, we procedurally generated many fin geometries using a scripting CAD, CADQuery.
This helped us to perform a comparison study among a huge number fin geometries.

\subsection{Optimization of microchannel topologies}
For laminar flow in confined channels, h scales inversely with channel width, making microscopic
channels desirable. Tuckerman and Pease \sidecite{1481851} showed that microchannel heat sinks have excellent heat transfer capacity and compact dimensions.
Hui Tan et al. \sidecite{TAN2019681} showed that the topology structure of microchannel had a significant impact on its heat transfer
performance, especially in high heat flux.
\\
Similar to fins, we procedurally generated many microchannel topologies and performed a comparison study.

\subsection{Nanofluids}
Nanofluids was coined by Choi\sidecite{osti_196525} , are engineered
colloids made up of a base fluid and the nanoparticles.
Nanoparticles have thermal conductivities, typically an order-of-magnitude
higher than those of the base fluids and with sizes significantly
smaller than 100 nm. The introduction of nanoparticles enhances
the heat transfer performance of the base fluids significantly. The
base fluids may be water, organic liquids (e.g. ethylene, triethylene-glycols, refrigerants, etc.), oils and lubricants, bio-fluids,
polymeric solutions and other common liquids. The nanoparticle
materials include chemically stable metals (e.g. gold, copper),
metal oxides (e.g. alumina, silica, zirconia, titania), oxide ceramics
(e.g. Al2O3, CuO), metal carbides (e.g. SiC), metal nitrides (e.g. AIN,
SiN), carbon in various forms (e.g. diamond, graphite, carbon
nanotubes, fullerene) and functionalized nanoparticles.
\\
We performed fluid-thermal coupling numerical simulation to assess heat transfer improvements due to the use of nanofluids together with optimized fin profiles and microchannel topology.
\end{document}
